\documentclass[12pt, article]{scrartcl}
\usepackage[english]{babel}
\usepackage{sectsty}
\allsectionsfont{\centering \normalfont\scshape}
\usepackage{fancyhdr}
\pagestyle{fancyplain}
\fancyhead{}
\fancyfoot[L]{}
\fancyfoot[C]{}
\fancyfoot[R]{\thepage}
\renewcommand{\headrulewidth}{0pt}
\renewcommand{\footrulewidth}{0pt}
\setlength{\headheight}{13.6pt}
\newcommand{\horrule}[1]{\rule{\linewidth}{#1}}

\title{	
\normalfont \normalsize 
\textsc{University of Idaho: CS 470 - Artificial Intelligence} \\ [25pt]
\horrule{0.5pt} \\[0.4cm]
\huge Project 1: Pathfinding\\
\horrule{2pt} \\[0.5cm]
}
\author{Andrew Schwartzmeyer}
\date{\normalsize\today}

\begin{document}
\maketitle % Print the title
\begin{abstract}
For this project I implemented a pathfinding program in Python to find the path
from specified start to goal coordinates on a rectangular ASCII map. The agent
is able to move up, down, left, and right, but not diagonally. I explored the
efficiency of multiple common search algorithms used to solve this problem,
including breadth-first, A* using Euclidean and Taxicab distance heuristics,
uniform cost, depth-first using a stack, recursive depth-first, and iterative
deepening.
\end{abstract}
\pagebreak

\section{The Breadth-first Algorithm}
Breadth-first is an algorithm for small problems only, as it has a time and
space complexity both of $O(b^{d})$. It is both complete and optimal, in that it
will find the solution (if one exists) and it will find the solution with the
fewest actions (but does not take cost into account, unless a sorted queue is
used).\\

The algorithm is simple: 
\begin{verbatim}
1.  Create a fringe list
2.  Queue root node (start state) onto fringe list
3.  while fringe list is not empty:
4.      Remove first state from list (do not pop last like a stack)
5.      if removed state is goal
6.          then return path
7.      else expand state and add new nodes to list
\end{verbatim}

The sample map is of size 15 by 20, with a start coordinate of (7, 0) and
goal coordinate of (7, 18). In the path and explored maps, `@' is the starting
point, `*' is the path, `\$' is the goal, and `\#' is an explored state. \\

\begin{tabular}{p{2in} p{2in} p{2in}}
Map & Path & Explored \\

\begin{verbatim}
MMMhhffffffffff
MMMMMhhffffffff
hMMMhhhffFFFfff
fhMhffFFFFFFFff
fhhhffFFFFFFFFF
ffffFFFFFFFFfff
rrrrfFFFFFFffff
fffrrffFFFfffff
RRffrrrfFFFFfff
fRffffrFFFFFFff
fRfffWWWWWFFFFF
fRffWWWWWWWWFFF
fRRfffWWWWWWWrr
ffRRRRffffWWfff
fffffRRRfffffff
fffffffRfffffff
hffffffRRRRRRRR
Mhhffffffffffff
Mhhffffffffffff
MMhhhffffffffff
\end{verbatim}

&

\begin{verbatim}
MMMhhff@fffffff
MMMMMhh*fffffff
hMMMhhh*fFFFfff
fhMhffF*FFFFFff
fhhhffF*FFFFFFF
ffffFFF*FFFFfff
rrrrfFF*FFFffff
fffrrff*FFfffff
RRffrrr*FFFFfff
fRff****FFFFFff
fRf**WWWWWFFFFF
fRf*WWWWWWWWFFF
fRR*ffWWWWWWWrr
ffR*RRffffWWfff
fff*fRRRfffffff
fff*fffRfffffff
hff*fffRRRRRRRR
Mhh*fffffffffff
Mhh****$fffffff
MMhhhffffffffff
\end{verbatim}

&

\begin{verbatim}
#######@#######
#######*#######
#######*#######
#######*#######
#######*#######
#######*#######
#######*#######
#######*#######
#######*#######
####****#######
###**WWWWW#####
###*WWWWWWWW###
###*##WWWWWWW##
###*######WW###
###*###########
###*###########
###*###########
###*####ff#####
###****$fff####
#######fffff###
\end{verbatim}
\end{tabular}

\end{document}
